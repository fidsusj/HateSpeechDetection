\section*{Abstract}

Hate speech is a very present problem, especially in the context of social media, where people can anonymously post whatever they want. While hate speech is not prohibited in the United States due to the \textit{freedom of speech}, countries in the European Union can sue hate speech actors for offending the public order or human dignity. Automatic detection of hate speech is a very important tool to efficiently handle and moderate hate speech in social media. 
Based on two labeled datasets from previous papers, our work concentrates on manual feature extraction and utilizing conventional machine learning methods to perform hate speech detection. Our main research question was whether conventional machine learning methods combined with suitable features can outperform neural network based approaches. As a result we found, that with deliberate feature extraction and optimizing machine learning methods such as Decision Tree or SVM, the results are competitive with a deep neural network approach, but they do not outperform them. Furthermore, we invested time into analyzing feature importances and hate speech cha\-rac\-te\-ris\-tics, where we found that our learned unigrams and trigrams as well as sentiment-based polarity scores indicate hate in a post.
Overall, the performance we achieved with conventional machine learning methods was on par with our neural network baseline with a F1-score of 93\% and an accuracy of slightly under 90\%.
Some further investigations were done with oversampling and undersampling the dataset, as the underlying data was unbalanced.
