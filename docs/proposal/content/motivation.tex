\section{Motivation}
Legal implementations on handling hate speech is different from one country to another. While hate speech is not prohibited in the United States due to the \textit{freedom of speech}, other countries - especially in the European Union - can sue hate speech actors for either offending the public order or human dignity. While being able to prosecute actors in public without much effort, the internet and especially social media platforms provide an easy and anonymous way to practice hate speech without legal consequence enforcements. Several steps were taken to tackle hate speech online, one of them being the \textit{code of conduct} on countering illegal hate speech online, an initiative of the european commission in close collaboration with major IT companies like \textit{Facebook}, \textit{Microsoft}, \textit{Twitter} and \textit{YouTube} \cite{EuropeanCommission.20200622}. While respecting the freedom of speech, these companies commit to delete hate speech contributions within 24 hours of the initial deletion request. 

To further automize the process of detecting hate speech contributions, several text analytics approaches have been evaluated in the recent past. Many of them are using methods of \textit{natural language processing} and \textit{deep learning} for hate speech detection and rely on meaningful features being learned automatically by deep neural networks instead of using hand-crafted features. In this work, the boundaries of conventional machine learning approaches for hate speech detection should get evaluated including manual feature extraction and subsequent text classification of hate speech and non-hate speech documents. The data sets used in this work originate from \textit{Twitter} posts \cite{ThomasDavidson.2020} and contributions to the \textit{White Supremacy Forum} \cite{OnadeGibert.2020}. The result of this work should show which features work best for which classifier and which problems can be addressed with conventional machine learning methods and which not as opposed to deep learning approaches.

In the first part of this proposal the results of the research phase will be presented (\autoref{section:research}) before moving on to the conrete project description (\autoref{section:description}). 
