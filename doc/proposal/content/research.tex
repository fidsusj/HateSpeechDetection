\section{Research Topic Summary}

There are three aspects to consider, derived from the reseach on automatic hate speech detection:
\begin{enumerate}
	\item Raw dataset
	\item Feature extraction
	\item Machine Learning approach
\end{enumerate}

\noindent
The following sections introduce these three aspects.

\subsection{Raw dataset}
To obtain a dataset one can use an existing already labelled dataset (e.g. done by Watanabe, Bouazizi and Ohtsuki \cite{8292838}), or one could label the data by hand (e.g. made by Oriola and Kotz\'{e} \cite{8963960}).
Depending on the dataset the classes in which the data is classified can differ. Prominent classifications for datasets in the domain of hate speech are the binary classification (no hate speech, hate speech) and the ternary classification (clean, offensive, hate speech). 
% TODO: Beispieldatensaetze

Another aspect is the distribution between the different classes. Is the distribution equally among the different classes or not? In case of imbalanced distributions the machine learning approach can be less performant and accurate \cite{8963960}.


\subsection{Feature extraction}
There are two classes of machine learning approaches for detecting hate speech (neuronal network approaches, classical machine learning approaches). They are described in detail in the next chapter. One main difference between the two introduced approaches is the process of feature extraction. In neuronal network approaches the used features are learned automatically, whereas classical machine learning techniques require a manual feature ex\-trac\-tion process. This is highly based on text analytics.

The following list provides an overview of possible features and their technical methods from the area of text analytics. The list is clustered into the four categories based on Watanabe, Bouazizi and Ohtsuki \cite{8292838}. The possible text analytics approaches derive from the literature review of Fortuna and Nunes \cite{10.1145/3232676}.
\begin{itemize}
	\item \textbf{Sentiment-based features}: Is the tweet rather positive or negative? \newline
	% allow us to extract the polarity of the tweet
	Text analytics approaches: Dictionaries, Rule Based Approaches, Ob\-jec\-ti\-vi\-ty-Subjectivity of the Language, Declarations of Superiority of the Ingroup \cite{10.1145/3232676}
	\item \textbf{Semantic features}: Which parts of the tweet are emphasized? \newline
	% allow us to find any emphasized expression
	Text analytics approaches: TF-IDF, Part-of-speech, Profanity Windows, Lexical Syntactic Feature-based, Topic Classification, Template Based Strategy, Word Sense Disambiguation Techniques, Othering Language \cite{10.1145/3232676}
	\item \textbf{Unigram features}: Are there any specific words marking hate speech? \newline
	% allow us to detect any explicit form of hate speech
	Text analytics approaches: N-grams, Bag-of-words \cite{10.1145/3232676}
	\item \textbf{Pattern features}: Are there any specific patterns marking hate speech? \newline
	% allow the identification of any longer or implicit forms of hate speech
	Text analytics approaches: Part-of-speech, Dictionaries, Typed De\-pen\-den\-cies,Word Embeddings \cite{10.1145/3232676}
\end{itemize}


\subsection{Machine learning approach}

Current research in the area of hate speech detection is often built upon neuronal network techniques. Exemplary therefore are the works of Roy et al. \cite{9253658}, Setyadi, Nasrun and Setianingsih \cite{8712109} and Kapil, Ekbal and Das \cite{kapil2020investigating}, to name just a few examples. Among other metrics by considering the accuracy one can evaluate the success of a choosen approach. Roy et al. \cite{9253658} have reached accuracies up to 95\% for detecting whether a tweet is hateful or not.

Nevertheless even classical machine learning techniques such as Support Vector Machines (SVM) combined with text analytics methods are a promis\-ing approach.
Watanabe, Bouazizi and Ohtsuki \cite{8292838} used a decision tree and reached an accuracy of 87.4\%.

Furthermore the results of classical machine learning techniques can be used as reference value to evaluate the neuronal network approach. This was done by Roy et al. \cite{9253658}.
